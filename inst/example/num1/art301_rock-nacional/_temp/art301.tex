% Options for packages loaded elsewhere
\PassOptionsToPackage{unicode}{hyperref}
\PassOptionsToPackage{hyphens}{url}

  \PassOptionsToPackage{dvipsnames,svgnames,x11names}{xcolor}



%%%% Clase del documento y configuraciones generales de apariencia
\documentclass[
  %
  %
    a4paper,%
  %
  %
  %
    DIV=calc,%
  %
    abstract=true%
  ]{scrartcl}%

\usepackage{amsmath,amssymb}


\usepackage{iftex}

\ifPDFTeX                       % if pdflatex
  \usepackage[T1]{fontenc}
  \usepackage[utf8]{inputenc}
  \usepackage{textcomp}         % provide euro and other symbols
\else % if luatex or xetex
      \usepackage{unicode-math}   % this also loads fontspec
    \defaultfontfeatures{Scale=MatchLowercase}
  \defaultfontfeatures[\rmfamily]{Ligatures=TeX,Scale=1}
\fi

  
  \usepackage[]{Alegreya}

\ifPDFTeX\else    
  \usepackage{fontspec}
          
  
      \fi


\IfFileExists{upquote.sty}{\usepackage{upquote}}{}

\IfFileExists{microtype.sty}{
  \usepackage[]{microtype}
  \UseMicrotypeSet[protrusion]{basicmath} 
}{}

  \makeatletter
  \@ifundefined{KOMAClassName}{%    
    \IfFileExists{parskip.sty}{%
      \usepackage{parskip}
    }{
      \setlength{\parindent}{0pt}
      \setlength{\parskip}{6pt plus 2pt minus 1pt}}
  }{
    \KOMAoptions{parskip=half}}
  \makeatother


\usepackage{xcolor}





%%%% Configuración de tablas
  \usepackage{longtable,booktabs,array}
    \usepackage{calc} % for calculating minipage widths
  % Correct order of tables after \paragraph or \subparagraph
  \usepackage{etoolbox}
  \makeatletter
  \patchcmd\longtable{\par}{\if@noskipsec\mbox{}\fi\par}{}{}
  \makeatother
  % Allow footnotes in longtable head/foot
  \IfFileExists{footnotehyper.sty}{\usepackage{footnotehyper}}{\usepackage{footnote}}
  \makesavenoteenv{longtable}

%%%% Configuración de figuras
  \usepackage{graphicx}
  \makeatletter
  \def\maxwidth{\ifdim\Gin@nat@width>\linewidth\linewidth\else\Gin@nat@width\fi}
  \def\maxheight{\ifdim\Gin@nat@height>\textheight\textheight\else\Gin@nat@height\fi}
  \makeatother
  % Scale images if necessary, so that they will not overflow the page
  % margins by default, and it is still possible to overwrite the defaults
  % using explicit options in \includegraphics[width, height, ...]{}
  \setkeys{Gin}{width=\maxwidth,height=\maxheight,keepaspectratio}
  % Set default figure placement to htbp
  \makeatletter
  \def\fps@figure{htbp}
  \makeatother



\setlength{\emergencystretch}{3em} % prevent overfull lines
\providecommand{\tightlist}{%
  \setlength{\itemsep}{0pt}\setlength{\parskip}{0pt}}

  \setcounter{secnumdepth}{-\maxdimen} % remove section numbering




% Referencias con CSL
  % definitions for citeproc citations
  \NewDocumentCommand\citeproctext{}{}
  \NewDocumentCommand\citeproc{mm}{%
    \begingroup\def\citeproctext{#2}\cite{#1}\endgroup}
  \makeatletter
  % allow citations to break across lines
  \let\@cite@ofmt\@firstofone
  % avoid brackets around text for \cite:
    \def\@biblabel#1{}
    \def\@cite#1#2{{#1\if@tempswa , #2\fi}}
  \makeatother
  \newlength{\cslhangindent}
  \setlength{\cslhangindent}{1.5em}
  \newlength{\csllabelwidth}
  \setlength{\csllabelwidth}{3em}
  \newenvironment{CSLReferences}[2] % #1 hanging-indent, #2 entry-spacing
   {\begin{list}{}{%
    \setlength{\itemindent}{0pt}
    \setlength{\leftmargin}{0pt}
    \setlength{\parsep}{0pt}
    % turn on hanging indent if param 1 is 1
    \ifodd #1
      \setlength{\leftmargin}{\cslhangindent}
      \setlength{\itemindent}{-1\cslhangindent}
    \fi
    % set entry spacing
    \setlength{\itemsep}{#2\baselineskip}}}
  {\end{list}}
  \usepackage{calc}
  \newcommand{\CSLBlock}[1]{\hfill\break\parbox[t]{\linewidth}{\strut\ignorespaces#1\strut}}
  \newcommand{\CSLLeftMargin}[1]{\parbox[t]{\csllabelwidth}{\strut#1\strut}}
  \newcommand{\CSLRightInline}[1]{\parbox[t]{\linewidth - \csllabelwidth}{\strut#1\strut}}
  \newcommand{\CSLIndent}[1]{\hspace{\cslhangindent}#1}

  \ifLuaTeX
    \usepackage[bidi=basic]{babel}
  \else
    \usepackage[bidi=default]{babel}
  \fi
      \babelprovide[main,import]{spanish}
      
      \babelprovide[import]{english}
      % get rid of language-specific shorthands (see #6817):
  \let\LanguageShortHands\languageshorthands
  \def\languageshorthands#1{}


% Customizaciones generales %%%%%%%%%%%%%%%%%%%%%

%%%% CARGA DE PAQUETES
\usepackage[export]{adjustbox}

% Cargar paquetes de íconos 
\usepackage{fontawesome5}
\usepackage{ccicons}

% Carga `setspace' para separación entre lineas (si no fue cargado)
\IfPackageLoadedTF{setspace}{}{\usepackage{setspace}}
% Carga `graphicx' si no fue cargado (dado que no se usa entorno pandoc para figures)
\IfPackageLoadedTF{graphicx}{}{\usepackage{graphicx}}
% Carga `etoolbox' 
\IfPackageLoadedTF{etoolbox}{}{\usepackage{etoolbox}}

% Carga de paquetes y configuraciones pedidas por el documento


%%%% COLORES CUSTOMIZADOS

\definecolor{verde_orcid}{RGB}{166,206,57}

%%%% CITAS LARGAS

% Definir estilo de citas largas (achica el tamaño)
\IfPackageLoadedTF{relsize}{}{\usepackage{relsize}}
\AtBeginEnvironment{quote}{\smaller}% Step font down one size relative to current font.

%%%% TABLAS Y FGURAS

% Agregado de \source y \notes para titulos secundarios de las tablas y figuras
\usepackage{caption}
\captionsetup{format=plain,labelsep=period,singlelinecheck=false,justification=centerlast,
              font={small,singlespacing,sf},labelfont={sc,bf},width=0.8\textwidth}
\newcommand{\source}[1]{\vspace{-9pt}\caption*{\footnotesize{\textit{Fuente:} {#1}}}}
\newcommand{\notes}[1]{\vspace{-9pt}\caption*{\footnotesize{\textit{Notas:} {#1}}}}



% Cambia denominación de "Cuadro" a "Tabla"
\renewcommand{\spanishtablename}{Tabla}

% Cambiar fuente y tamaño de letra para entorno de tablas (table)
\AtBeginEnvironment{tabular}{\sffamily\footnotesize}
\IfPackageLoadedTF{tabularray}{%
\AtBeginEnvironment{tblr}{\sffamily\footnotesize}}{}

%%%% ENCABEZADO DE PÁGINA
% Si utiliza clases de Koma-script debe utilizar el paquete 'scrlayer-scrpage'
% Si utiliza las clases estándar debe usar el paquete 'fancyhdr'
\usepackage[headsepline]{scrlayer-scrpage}
\usepackage[breakwords]{truncate}

\setkomafont{pagehead}{\normalfont\sffamily\small}

\ihead{\textit{\truncate{0.45\textwidth}{Lorem ipsum}}}
\ohead{\textbf{\textasciitilde!guri\_ \{An example journal\}}}

\setkomafont{pagefoot}{\normalfont\footnotesize}
\ifoot[\rule{0.4\textwidth}{0.5pt} \vskip 0.5em Publicación de \emph{guri}.
      ISSN-e: 0001-2000.\\
      Esta obra está bajo una licencia Creative Commons
Attribution-NonCommercial-ShareAlike 4.0 International
License (https://creativecommons.org/licenses/by-nc/4.0/).]{}
\cfoot[]{\pagemark}

% Definir primera página del artícul (si se utiliza paginación personalizada)

%%%% PÁGINA DE TÍTULO 
\setkomafont{titlehead}{\raggedright\sffamily\small}
\setkomafont{title}{\raggedright\sffamily}
\setkomafont{author}{\raggedright\sffamily\setlength{\tabcolsep}{0pt}}
\setkomafont{publishers}{\raggedright\sffamily\footnotesize}
\setkomafont{dedication}{\raggedright\sffamily\small}

\titlehead{%
  \textbf\textbf{DOSSIER}\hfill\textbf{\textasciitilde!guri\_ \{An
example journal\}}\\%
  \vspace{4pt}
  \hfill Vol. 10 Núm. 1 (2023)
  \\ \hfill DOI: \href{https://doi.org/10.54/242423423.dsa211}{10.54/242423423.dsa211}%
 }%

\title{Lorem ipsum}

  \subtitle{Duis aute irure dolor\\ \vspace{1em} \textit{English lorem
ipsum}}

\author{{\begin{tabular}[l]{@{}l@{}}%
  \large{Carlos
Alberto {García}}\textsuperscript{a;b} \href{https://orcid.org/0000-0000-1234-5678}{\textcolor{verde_orcid}{\faOrcid}} \\ \large{Luis
Alberto {Spinetta}}\textsuperscript{a;c} \href{https://orcid.org/0000-0005-0901-8282}{\textcolor{verde_orcid}{\faOrcid}}
\end{tabular}}%
}
    
\date{}

\publishers{
  \vspace{1em}
  \textsuperscript{a} Rock Nacional, Argentina.\\\textsuperscript{b} Say
No More, Argentina.\\\textsuperscript{c} Alma de diamante, Argentina.
  \\ \vspace{1em} \textit{Recibido:} 1 de Septiembre de 2023; \textit{Aceptado:} 11 de Noviembre de 2023.\\%
}




%%% Credit, Ack and app Enviroment

\newenvironment{Credit}[1][Declaración de contribuciones de autoría (CRediT)]
    {\subsection*{#1}
    \sffamily\small}

\newenvironment{Ack}{\begin{Credit}[Agradecimiento]}{\end{Credit}}%

\newenvironment{app}{\newpage}{}
\ifLuaTeX
  \usepackage{selnolig}  % disable illegal ligatures
\fi






\usepackage{bookmark}

\IfFileExists{xurl.sty}{\usepackage{xurl}}{} % add URL line breaks if available
\urlstyle{same}



\hypersetup{
      pdftitle={Lorem ipsum},
        pdfauthor={Carlos Alberto García, Luis Alberto Spinetta},
    pdfsubject={\textasciitilde!guri\_ \{An example
journal\} - Volume: 10  - Issue: 1  (2023) a301 doi: 10.54/242423423.dsa211},
      pdflang={es-ES},
        pdfkeywords={kw1\_es, kw2\_es},
    pdfinfo={
          DOI={10.54/242423423.dsa211},
              abstract={Lorem ipsum dolor sit amet, consectetur
adipiscing elit, sed do eiusmod tempor incididunt ut labore et dolore
magna aliqua. Ut enim ad minim veniam, quis nostrud exercitation ullamco
laboris nisi ut aliquip ex ea commodo consequat. Duis aute irure dolor
in reprehenderit in voluptate velit esse cillum dolore eu fugiat nulla
pariatur. Excepteur sint occaecat cupidatat non proident, sunt in culpa
qui officia deserunt mollit anim id est laborum.},
        journal=\textasciitilde!guri\_ \{An example journal\}
      },
  % Define color de links
      colorlinks=true,
    linkcolor={CadetBlue},
    filecolor={CadetBlue},
    citecolor={CadetBlue},
    urlcolor={CadetBlue},
  pdfcreator={LaTeX via pandoc \& \textasciitilde!guri\_}}

%%%% Inicio de documento

\begin{document}
  
  
  \maketitle

  
    
    
    \begin{abstract}

      \noindent{Lorem ipsum dolor sit amet, consectetur adipiscing elit,
sed do eiusmod tempor incididunt ut labore et dolore magna aliqua. Ut
enim ad minim veniam, quis nostrud exercitation ullamco laboris nisi ut
aliquip ex ea commodo consequat. Duis aute irure dolor in reprehenderit
in voluptate velit esse cillum dolore eu fugiat nulla pariatur.
Excepteur sint occaecat cupidatat non proident, sunt in culpa qui
officia deserunt mollit anim id est laborum.}

      \nopagebreak

      \noindent\sffamily\textit{Palabras claves: kw1\_es; kw2\_es}.%

    \end{abstract}
    
        \begin{otherlanguage}{english}

      \begin{abstract}      
 
        \noindent{Maecenas risus nisl, vehicula a molestie eget, varius
ut quam. Vestibulum quis blandit metus, at fringilla magna. Morbi
finibus dui lacus, viverra faucibus dolor laoreet ac. Praesent vehicula
neque tempor lacus scelerisque, id tincidunt nulla condimentum. Etiam
convallis lobortis nulla ut fringilla. Phasellus aliquet euismod mauris
ac commodo. Nam dictum, nunc sed tempus dictum, sapien tortor commodo
tortor, ac fermentum lacus metus nec ex. Nunc ac finibus est. Integer ut
vestibulum ante.}

        \nopagebreak

        \noindent\sffamily\textit{Keywords: kw1\_en; kw2\_en}.%

      \end{abstract}
    \end{otherlanguage}
      

    
    
      
  
  
  \emph{Este informe presenta las anotaciones derivadas del Taller de
  justicia ambiental y justicia climática realizado el junio 6 de 2022
  en la Ciudad de México por la doctora Gabriela Merlinsky. Para ello se
  realiza un resumen de características y problemas generales de la
  ecología política y luego la propuesta de seis aspectos de estudio.}

  En la ecología política se encuentran perspectivas situadas, alianzas
  de producción de conocimientos de actores en los territorios que
  producen conocimientos, epidemiologías populares, como cuando las
  comunidades hacen sus propias investigaciones o piden ayuda a algún
  grupo de investigación de las universidades para medir, por ejemplo,
  la calidad del agua ante la duda y alerta del crecimiento de cáncer en
  las comunidades cercanas, (mientras que otros actores dicen que las
  enfermedades son la responsabilidad de las comunidades por sus malos
  hábitos de higiene, usar pisos o elementos que contienen plomo, etc.,
  y se culpa a las comunidades mismas. Entonces ellas acuden a un
  ejercicio de epidemiología popular).

  También están las cargas desproporcionadas de sitios de contaminación,
  rellenos sanitarios, sitios de desechos peligrosos o tóxicos que se
  ubican en cierto tipo de comunidades, y no en otras, pero que en
  realidad no deberían estar ubicados en ninguna comunidad, la
  visibilidad en la producción del conocimiento y, por consiguiente, la
  visibilidad de los cuerpos, de los conocimientos alternativos, frente
  a las concepciones del ambiente y las pruebas judiciales como únicas
  que se aceptan, mapeos de actores y sitios específicos, cartografías
  sociales, metodologías alternativas, investigación acción
  participativa, entre otras.

  La ecología política latinoamericana actúa bajo el supuesto según el
  cual ha habido colonialidad de la naturaleza, como algo de dominio de
  la ciencia y de la producción económica, como objeto de disputa,
  apropiación, resignificación. Sin embargo, se ha estado dando un giro
  ontológico a partir de las voces de los pueblos andinos, en especial
  las indígenas, las ontologías relacionales. Por ejemplo, para el
  cuidado de una montaña, más que considerarla como una creencia en un
  ser idealizado, se entiende que ella posee propiedades agentivas,
  agencia propia, lo que no es fácil de entender desde contextos urbanos
  y concepciones occidentales.

  Toda ecología es política. Los debates sobre el clima, los trabajos
  académicos del sur global están emergiendo en estas temáticas, pero
  todavía se ve invisibilización del sur global. Entonces se requiere
  politizar varias categorías, diferentes formas de jerarquía que,
  basadas en la clase, etnia y género, están en la base de la
  desigualdad. Se requiere mirar sus implicaciones políticas y atender a
  los movimientos de justicia ambiental.

  En la ecología política hay varios aspectos a considerar: las
  desigualdades y entre ellas las desigualdades ambientales, las formas
  de interseccionalidad de distintas jerarquías (clase, etnia, género),
  y las exclusiones en las prácticas políticas. Así, la ecología
  política estudia algunas constantes en los conflictos: las disputas,
  los actores, el escalamiento o re-escalamiento y las exclusiones.
  Acerca de esto último, las comunidades se quejan por la falta de
  licencia social o el derecho a ser consultadas en las decisiones que
  les afectan, como en los casos para explotación minero-energética. En
  efecto, muchas de estas decisiones aun siendo legales no son legítimas
  ante la comunidad. Existe desigualdad y también una diferenciación en
  el concepto de ambiente que incluye disputas de sentido por el
  problema ambiental.

  Se presentan seis aspectos fundamentales en la ecología política:

  \begin{enumerate}
  \def\labelenumi{\arabic{enumi}.}
  \item
    La discusión debe ser tangible no abstracta. En efecto, se está
    hablando del derecho a la vida, a la salud, a la tierra, a las
    semillas, a lo agroecológico, a lo cotidiano. Más que partes por
    millón, gases efecto invernadero, emisiones de carbono, toxicidad,
    más que otros conceptos técnicos o elementos, se trata de la vida y
    vivencia tangible. No obstante, hay un clasismo en la discusión en
    esos términos abstractos. Ha habido expropiación y despojo no solo
    cultural, económico, social, sino también epistémico. Se requieren
    agendas que construyan, a partir de modos tangibles, la discusión de
    las cuestiones claves para la vida. Hay que escuchar cómo "somos una
    clase social que solo tiene su trabajo o su no trabajo". La
    desigualdad es fuerte.
  \item
    La inconmensurabilidad de los valores (\citeproc{ref-3}{Martínez
    Alier, 1998}). En el estudio de la economía ecológica, este autor
    analiza los lenguajes de valoración. Si bien algunos colectivos o
    comunidades entienden que es mejor recibir alguna compensación
    económica que ninguna, lo que no están dispuestos a realizar es
    traducir las cuestiones fundamentales o lo que está en juego al
    lenguaje del dinero. ¿Cuál es el valor de un paisaje? ¿Cuánto vale
    la vida? ¿Cuánto vale la salud no traducido al dinero? Es así como
    hay otros lenguajes que no son económicos y deben estar en el centro
    de la discusión. (Aunque no recibir nada es lo peor que puede
    pasarles a las comunidades en un conflicto socio-ambiental).
  \item
    El discurso experto no es el que rige a la discusión. La persona
    experta y el discurso experto en ciencias naturales no son de suyo
    quienes deban cerrar la discusión ambiental. En efecto, puede
    recurrirse a esta en ayuda para determinar con su ciencia algunos
    aspectos, pero ella no tiene la última palabra en la discusión en
    términos del bagaje del lenguaje de la vida y las tensiones que se
    manejan. Sin embargo, lo que se ha visto es que el lenguaje
    económico es el que ha estado prefijado para los debates, por encima
    del científico, y más aún, por encima del lenguaje de la vida.
  \item
    La cuestión del reconocimiento y la discriminación. ¿A quién le toca
    la mayor carga ambiental? Están las mujeres, los grupos y
    comunidades locales y ancestrales, las clases sociales populares y,
    especialmente, las comunidades de personas recicladoras. Existe una
    paradoja de no querer la basura en un territorio propio, pero si se
    pone en otro lado no importa. Sin embargo, no tendría que estar en
    ningún lugar "en el patio trasero de nadie". En efecto, hay que
    pensar otros modelos de reducción y de disposición final.
  \item
    La violencia simbólica y la discriminación, según la cual a quienes
    les ha tocado la mayor carga ambiental ya han sido discriminados
    anteriormente por otras razones. Está el caso de las mujeres que
    denuncian: "Trabajamos con la basura, pero no queremos ser tratadas
    así". Sea como fuere, se debe incorporar la cuestión social, étnica
    y de género a la ambiental.
  \item
    La necesidad y demanda de políticas estatales. Es decir, antes de
    que sucedan los acontecimientos, el Estado debió impedir, regular,
    proponer alternativas de solución del problema, inspeccionar,
    supervisar. También hay que tener en cuenta las demandas de las
    organizaciones ambientalistas cuando dicen: "ustedes están
    preocupados por las especies en extinción. ¡Nosotros somos
    comunidades en extinción!", porque hay exposición al peligro.
    Martínez Alier, en su atlas de la justicia ambiental
    {[}\href{https://ejatlas.org/?translate=es}{LINK}{]} incluye a
    varios países en el estudio progresivo de la cuestión.
  \end{enumerate}

  Vivir bien es un concepto interpelado, muy profundo. El Buen Vivir,
  \emph{sumak kawsay} (Ecuador), \emph{suma qamaña} (Bolivia), señalan
  que no se trata de crecer económicamente con modelos foráneos
  impuestos, o de tener o explotar más recursos, sino de vivir bien. Sin
  embargo, está la disputa por el conocimiento, la necesidad de ir desde
  la seguridad hacia la soberanía alimentaria, el hecho de que no se
  trata de exportar commodities y seguir explotando hasta la última
  cantidad de algún elemento. Hay varias concepciones como los derechos
  de la naturaleza, la biopiratería según la cual se ha dado la
  expropiación y apropiación de recursos genéticos y derechos de
  propiedad de los pueblos indígenas, el agua como derecho humano
  (frente a la privatización del agua que considera el agua como
  mercancía). También se requiere poner atención a los jóvenes y a sus
  activismos urbanos, mirar hacia las personas recicladoras o cartoneras
  que trabajan con la basura en las calles y en los depósitos por cuanto
  tienen un rol central para la sustentabilidad.

  Se requiere igualmente de agendas prospectivistas y
  post-extractivistas en lo rural como en las zonas periféricas de las
  grandes metrópolis como, por ejemplo, en Córdoba, Argentina, donde las
  comunidades cercanas reciben los agrotóxicos esparcidos para el
  cultivo industrial de la soja.

  Finalmente, existe un gran potencial performativo en el concepto de
  justicia climática al conectar sitios dispares del mundo y
  vincularlos. Esto permite analizar cómo los países del sur global son
  los que menos emiten contaminaciones (¿quiénes emiten las basuras?
  ¿cuánta y a dónde van a parar?), las discusiones por la justicia ante
  el capitalismo fósil y carbonífero, la desigualdad entre el norte y el
  sur global, el enfrentamiento a las élites, las definiciones de
  justicia localizada para cada comunidad, las diferencias entre los
  movimientos o luchas ambientales y climáticas, y las medidas de
  reparación.

  \begin{Credit}

  \phantomsection\label{Credit}
  \textbf{García:} Conceptualización (Conceptualization); Análisis
  formal (Formal Analysis); Adquisición de Financiamiento (Funding
  acquisition); Investigación (Investigation); Administración de
  proyecto (Project administration); Recursos (Resources); Validación
  (Validation); Redacción - preparación del borrador original (Writing
  -- original draft); Redacción - revisión y edición (Writing -- review
  \& editing). \textbf{Spinetta:} Conceptualización (Conceptualization);
  Análisis formal (Formal Analysis); Adquisición de Financiamiento
  (Funding acquisition); Investigación (Investigation); Administración
  de proyecto (Project administration); Recursos (Resources); Validación
  (Validation); Redacción - preparación del borrador original (Writing
  -- original draft); Redacción - revisión y edición (Writing -- review
  \& editing).

  \end{Credit}

  \begin{Ack}

  \phantomsection\label{Ack}
  Porta phasellus natoque fusce tempus aptent sem fames, mattis
  suspendisse inceptos ligula accumsan taciti.

  \end{Ack}

  \section{Referencias
  bibliográficas}\label{referencias-bibliogruxe1ficas}

  \phantomsection\label{refs}
  \begin{CSLReferences}{1}{0}
  \bibitem[\citeproctext]{ref-3}
  Martínez Alier, J. (1998). \emph{La economía ecológica como ecología
  humana}. Fundación César Manrique.

  \end{CSLReferences}

    
  %%%% Referencias bibliográficas
  
  
    
\end{document}
