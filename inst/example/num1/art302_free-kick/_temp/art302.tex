% Options for packages loaded elsewhere
\PassOptionsToPackage{unicode}{hyperref}
\PassOptionsToPackage{hyphens}{url}

  \PassOptionsToPackage{dvipsnames,svgnames,x11names}{xcolor}



%%%% Clase del documento y configuraciones generales de apariencia
\documentclass[
  %
  %
    a4paper,%
  %
  %
  %
    DIV=calc,%
  %
    abstract=true%
  ]{scrartcl}%

\usepackage{amsmath,amssymb}


\usepackage{iftex}

\ifPDFTeX                       % if pdflatex
  \usepackage[T1]{fontenc}
  \usepackage[utf8]{inputenc}
  \usepackage{textcomp}         % provide euro and other symbols
\else % if luatex or xetex
      \usepackage{unicode-math}   % this also loads fontspec
    \defaultfontfeatures{Scale=MatchLowercase}
  \defaultfontfeatures[\rmfamily]{Ligatures=TeX,Scale=1}
\fi

  
  \usepackage[]{Alegreya}

\ifPDFTeX\else    
  \usepackage{fontspec}
          
  
      \fi


\IfFileExists{upquote.sty}{\usepackage{upquote}}{}

\IfFileExists{microtype.sty}{
  \usepackage[]{microtype}
  \UseMicrotypeSet[protrusion]{basicmath} 
}{}

  \makeatletter
  \@ifundefined{KOMAClassName}{%    
    \IfFileExists{parskip.sty}{%
      \usepackage{parskip}
    }{
      \setlength{\parindent}{0pt}
      \setlength{\parskip}{6pt plus 2pt minus 1pt}}
  }{
    \KOMAoptions{parskip=half}}
  \makeatother


\usepackage{xcolor}





%%%% Configuración de tablas
  \usepackage{longtable,booktabs,array}
    \usepackage{calc} % for calculating minipage widths
  % Correct order of tables after \paragraph or \subparagraph
  \usepackage{etoolbox}
  \makeatletter
  \patchcmd\longtable{\par}{\if@noskipsec\mbox{}\fi\par}{}{}
  \makeatother
  % Allow footnotes in longtable head/foot
  \IfFileExists{footnotehyper.sty}{\usepackage{footnotehyper}}{\usepackage{footnote}}
  \makesavenoteenv{longtable}

%%%% Configuración de figuras
  \usepackage{graphicx}
  \makeatletter
  \def\maxwidth{\ifdim\Gin@nat@width>\linewidth\linewidth\else\Gin@nat@width\fi}
  \def\maxheight{\ifdim\Gin@nat@height>\textheight\textheight\else\Gin@nat@height\fi}
  \makeatother
  % Scale images if necessary, so that they will not overflow the page
  % margins by default, and it is still possible to overwrite the defaults
  % using explicit options in \includegraphics[width, height, ...]{}
  \setkeys{Gin}{width=\maxwidth,height=\maxheight,keepaspectratio}
  % Set default figure placement to htbp
  \makeatletter
  \def\fps@figure{htbp}
  \makeatother



\setlength{\emergencystretch}{3em} % prevent overfull lines
\providecommand{\tightlist}{%
  \setlength{\itemsep}{0pt}\setlength{\parskip}{0pt}}

  \setcounter{secnumdepth}{-\maxdimen} % remove section numbering




% Referencias con CSL

  \ifLuaTeX
    \usepackage[bidi=basic]{babel}
  \else
    \usepackage[bidi=default]{babel}
  \fi
      \babelprovide[main,import]{spanish}
      
      \babelprovide[import]{english}
      % get rid of language-specific shorthands (see #6817):
  \let\LanguageShortHands\languageshorthands
  \def\languageshorthands#1{}


% Customizaciones generales %%%%%%%%%%%%%%%%%%%%%

%%%% CARGA DE PAQUETES
\usepackage[export]{adjustbox}

% Cargar paquetes de íconos 
\usepackage{fontawesome5}
\usepackage{ccicons}

% Carga `setspace' para separación entre lineas (si no fue cargado)
\IfPackageLoadedTF{setspace}{}{\usepackage{setspace}}
% Carga `graphicx' si no fue cargado (dado que no se usa entorno pandoc para figures)
\IfPackageLoadedTF{graphicx}{}{\usepackage{graphicx}}
% Carga `etoolbox' 
\IfPackageLoadedTF{etoolbox}{}{\usepackage{etoolbox}}

% Carga de paquetes y configuraciones pedidas por el documento


%%%% COLORES CUSTOMIZADOS

\definecolor{verde_orcid}{RGB}{166,206,57}

%%%% CITAS LARGAS

% Definir estilo de citas largas (achica el tamaño)
\IfPackageLoadedTF{relsize}{}{\usepackage{relsize}}
\AtBeginEnvironment{quote}{\smaller}% Step font down one size relative to current font.

%%%% TABLAS Y FGURAS

% Agregado de \source y \notes para titulos secundarios de las tablas y figuras
\usepackage{caption}
\captionsetup{format=plain,labelsep=period,singlelinecheck=false,justification=centerlast,
              font={small,singlespacing,sf},labelfont={sc,bf},width=0.8\textwidth}
\newcommand{\source}[1]{\vspace{-9pt}\caption*{\footnotesize{\textit{Fuente:} {#1}}}}
\newcommand{\notes}[1]{\vspace{-9pt}\caption*{\footnotesize{\textit{Notas:} {#1}}}}



% Cambia denominación de "Cuadro" a "Tabla"
\renewcommand{\spanishtablename}{Tabla}

% Cambiar fuente y tamaño de letra para entorno de tablas (table)
\AtBeginEnvironment{tabular}{\sffamily\footnotesize}
\IfPackageLoadedTF{tabularray}{%
\AtBeginEnvironment{tblr}{\sffamily\footnotesize}}{}

%%%% ENCABEZADO DE PÁGINA
% Si utiliza clases de Koma-script debe utilizar el paquete 'scrlayer-scrpage'
% Si utiliza las clases estándar debe usar el paquete 'fancyhdr'
\usepackage[headsepline]{scrlayer-scrpage}
\usepackage[breakwords]{truncate}

\setkomafont{pagehead}{\normalfont\sffamily\small}

\ihead{\textit{\truncate{0.45\textwidth}{Ejecución de Tiros Libres en
Fútbol}}}
\ohead{\textbf{\textasciitilde!guri\_ \{An example journal\}}}

\setkomafont{pagefoot}{\normalfont\footnotesize}
\ifoot[\rule{0.4\textwidth}{0.5pt} \vskip 0.5em Publicación de \emph{guri}.
      ISSN-e: 0001-2000.\\
      Esta obra está bajo una licencia Creative Commons
Attribution-NonCommercial-ShareAlike 4.0 International
License (https://creativecommons.org/licenses/by-nc/4.0/).]{}
\cfoot[]{\pagemark}

% Definir primera página del artícul (si se utiliza paginación personalizada)

%%%% PÁGINA DE TÍTULO 
\setkomafont{titlehead}{\raggedright\sffamily\small}
\setkomafont{title}{\raggedright\sffamily}
\setkomafont{author}{\raggedright\sffamily\setlength{\tabcolsep}{0pt}}
\setkomafont{publishers}{\raggedright\sffamily\footnotesize}
\setkomafont{dedication}{\raggedright\sffamily\small}

\titlehead{%
  \textbf\textbf{ARTÍCULOS}\hfill\textbf{\textasciitilde!guri\_ \{An
example journal\}}\\%
  \vspace{4pt}
  \hfill Vol. 10 Núm. 1 (2023)
  \\ \hfill DOI: \href{https://doi.org/10.1177/004209802311733}{10.1177/004209802311733}%
 }%

\title{Ejecución de Tiros Libres en Fútbol}

  \subtitle{Un Análisis Científico\\ \vspace{1em} \textit{Free Kicks in
Football. A Scientific Analysis}}

\author{{\begin{tabular}[l]{@{}l@{}}%
  \large{Diego
Armando {Maradona}}\textsuperscript{a} \href{https://orcid.org/0000-0010-0010-0010}{\textcolor{verde_orcid}{\faOrcid}} \\ \large{Lionel {Messi}}\textsuperscript{a} \href{https://orcid.org/0000-0001-2345-6789}{\textcolor{verde_orcid}{\faOrcid}}
\end{tabular}}%
}
    
\date{}

\publishers{
  \vspace{1em}
  \textsuperscript{a} Selección de Futbol Argentina, Argentina.
  %
}




%%% Credit, Ack and app Enviroment

\newenvironment{Credit}[1][Declaración de contribuciones de autoría (CRediT)]
    {\subsection*{#1}
    \sffamily\small}

\newenvironment{Ack}{\begin{Credit}[Agradecimiento]}{\end{Credit}}%

\newenvironment{app}{\newpage}{}
\ifLuaTeX
  \usepackage{selnolig}  % disable illegal ligatures
\fi






\usepackage{bookmark}

\IfFileExists{xurl.sty}{\usepackage{xurl}}{} % add URL line breaks if available
\urlstyle{same}



\hypersetup{
      pdftitle={Ejecución de Tiros Libres en Fútbol},
        pdfauthor={Diego Armando Maradona, Lionel Messi},
    pdfsubject={\textasciitilde!guri\_ \{An example
journal\} - Volume: 10  - Issue: 1  (2023) 20201 doi: 10.1177/004209802311733},
      pdflang={es-ES},
      pdfinfo={
          DOI={10.1177/004209802311733},
            journal=\textasciitilde!guri\_ \{An example journal\}
      },
  % Define color de links
      colorlinks=true,
    linkcolor={CadetBlue},
    filecolor={CadetBlue},
    citecolor={CadetBlue},
    urlcolor={CadetBlue},
  pdfcreator={LaTeX via pandoc \& \textasciitilde!guri\_}}

%%%% Inicio de documento

\begin{document}
  
  
  \maketitle

  
  

    
    
      
  
  
  \section{Introducción}\label{introducciuxf3n}

  El fútbol es uno de los deportes más populares y apreciados en todo el
  mundo. Entre las numerosas habilidades que un jugador de fútbol debe
  dominar, la ejecución de tiros libres se destaca como un aspecto
  crucial del juego. La capacidad de convertir un tiro libre en un gol
  puede cambiar el curso de un partido y llevar a la victoria a un
  equipo. En este artículo, exploraremos los aspectos científicos detrás
  de la ejecución de tiros libres en fútbol, basándonos en
  investigaciones previas publicadas en revistas científicas
  especializadas.

  \section{Mecánica de la Ejecución de Tiros
  Libres}\label{mecuxe1nica-de-la-ejecuciuxf3n-de-tiros-libres}

  La ejecución de tiros libres en el fútbol es una habilidad fundamental
  que requiere una comprensión profunda de la mecánica detrás de cada
  disparo. La mecánica adecuada de la ejecución de tiros libres en
  fútbol ha sido objeto de numerosos estudios científicos. Según el
  trabajo de García et al.~(2019) en la "Revista de Biomecánica
  Deportiva", la biomecánica de un tiro libre exitoso se basa en una
  combinación de factores, incluyendo la posición del pie de apoyo, el
  ángulo de inclinación del cuerpo, la velocidad de la pelota y la
  dirección de la mirada del jugador.

  \begin{enumerate}
  \def\labelenumi{\arabic{enumi}.}
  \item
    \emph{Posición del Pie de Apoyo:} La posición del pie de apoyo es
    esencial para garantizar la estabilidad y la precisión en el momento
    del disparo. Un pie de apoyo colocado correctamente permite al
    jugador mantener el equilibrio y generar la potencia adecuada para
    el tiro.
  \item
    \emph{Ángulo de Inclinación del Cuerpo:} El ángulo de inclinación
    del cuerpo del jugador también influye en la trayectoria de la
    pelota. Un ángulo de inclinación adecuado puede determinar si la
    pelota pasa por encima de la barrera o si se curva alrededor de
    ella.
  \item
    \emph{Velocidad de la Pelota:} La velocidad con la que se golpea la
    pelota es un factor crítico. Investigaciones sugieren que la
    velocidad adecuada puede evitar que el portero tenga tiempo
    suficiente para reaccionar y bloquear el tiro (García et al., 2019).
  \item
    \emph{Dirección de la Mirada del Jugador:} La dirección de la mirada
    del jugador influye en la precisión del disparo. Fijar la mirada en
    un punto específico de la portería puede ayudar al jugador a apuntar
    con mayor precisión.
  \end{enumerate}

  \section{Influencia de la Superficie del
  Campo}\label{influencia-de-la-superficie-del-campo}

  La superficie del campo de juego también ha sido objeto de
  investigación en relación con la ejecución de tiros libres. En un
  estudio reciente publicado en la "Revista de Ciencias del Deporte y la
  Tecnología" (Smith y Johnson, 2021), se analizó cómo la superficie del
  campo afecta la trayectoria de la pelota durante un tiro libre. Los
  resultados sugieren que la superficie del campo puede influir
  significativamente en la distancia y la curvatura de la pelota.

  El campo de juego en el fútbol no siempre es uniforme, y la superficie
  del campo puede variar. Investigaciones recientes, como el estudio de
  Smith y Johnson (2021) en la "Revista de Ciencias del Deporte y la
  Tecnología", han examinado cómo la superficie del campo afecta la
  ejecución de tiros libres.

  La superficie del campo puede influir en la distancia y la curvatura
  de la pelota durante un tiro libre. Superficies más duras pueden
  permitir que la pelota ruede más rápido y se mantenga más baja,
  mientras que superficies más blandas pueden afectar la velocidad y la
  trayectoria de la pelota. Comprender cómo la superficie del campo
  interactúa con la mecánica de los tiros libres es fundamental para los
  jugadores y entrenadores.

  \section{Entrenamiento Específico}\label{entrenamiento-especuxedfico}

  El entrenamiento específico para tiros libres es un tema importante en
  la literatura científica. En un artículo de revisión de la "Revista de
  Entrenamiento Deportivo Avanzado" (Martínez, 2018), se destacó la
  importancia de la práctica constante y la retroalimentación para
  mejorar la precisión y la potencia de los tiros libres. Además, se
  sugiere que el entrenamiento de la concentración mental puede aumentar
  la efectividad de los tiros libres, como se menciona en el estudio de
  López et al.~(2020) en la "Revista de Psicología del Deporte".

  El entrenamiento específico para tiros libres es una parte esencial
  del desarrollo de las habilidades de un jugador. Como se destaca en el
  artículo de revisión de Martínez (2018) en la "Revista de
  Entrenamiento Deportivo Avanzado", el entrenamiento de tiros libres
  debe ser meticuloso y enfocado en aspectos específicos.

  Se recomienda la práctica constante y la retroalimentación para
  mejorar la precisión y la potencia de los tiros libres. Además, el
  entrenamiento de la concentración mental es una estrategia importante
  para aumentar la efectividad de los tiros libres, como se menciona en
  el estudio de López et al.~(2020) en la "Revista de Psicología del
  Deporte".

  \section{Factores Psicológicos}\label{factores-psicoluxf3gicos}

  La ejecución de tiros libres en situaciones de alta presión durante un
  partido también ha sido objeto de investigación psicológica. En el
  artículo de la "Revista de Psicología del Rendimiento Deportivo" de
  Rodríguez (2017), se exploran las estrategias mentales utilizadas por
  los jugadores para mantener la concentración y la calma durante un
  tiro libre importante.

  Los factores psicológicos desempeñan un papel crucial en la ejecución
  de tiros libres, especialmente en situaciones de alta presión durante
  un partido. En el artículo de la "Revista de Psicología del
  Rendimiento Deportivo" de Rodríguez (2017), se exploran las
  estrategias mentales utilizadas por los jugadores para mantener la
  concentración y la calma durante un tiro libre importante.

  Los aspectos psicológicos, como la autoconfianza, la visualización y
  la gestión del estrés, son fundamentales para garantizar un
  rendimiento óptimo en situaciones de tiro libre.

  \section{Conclusiones}\label{conclusiones}

  En conclusión, la ejecución de tiros libres en fútbol es un tema
  complejo que involucra aspectos biomecánicos, físicos y psicológicos.
  La investigación científica en este campo ha proporcionado valiosos
  conocimientos sobre la mecánica, la influencia de la superficie del
  campo, el entrenamiento y los factores psicológicos que influyen en la
  precisión y efectividad de los tiros libres. Estos estudios han
  contribuido significativamente al desarrollo y la mejora de las
  habilidades de ejecución de tiros libres en el fútbol, y continúan
  siendo una fuente importante de información para entrenadores y
  jugadores en todo el mundo.

  En resumen, la ejecución de tiros libres en fútbol es una habilidad
  compleja que involucra aspectos biomecánicos, físicos y psicológicos.
  La investigación científica en este campo ha proporcionado
  conocimientos valiosos sobre la mecánica, la influencia de la
  superficie del campo, el entrenamiento específico y los factores
  psicológicos que influyen en la precisión y efectividad de los tiros
  libres. Estos estudios han contribuido significativamente al
  desarrollo y la mejora de las habilidades de ejecución de tiros libres
  en el fútbol y continúan siendo una fuente importante de información
  para entrenadores y jugadores en todo el mundo.

  \begin{Credit}

  \phantomsection\label{Credit}
  \textbf{Maradona:} Conceptualización (Conceptualization); Adquisición
  de Financiamiento (Funding acquisition); Recursos (Resources);
  Validación (Validation); Redacción - preparación del borrador original
  (Writing -- original draft); Redacción - revisión y edición (Writing
  -- review \& editing). \textbf{Messi:} Curación de datos (Data
  curation); Recursos (Resources); Visualización (Visualization).

  \end{Credit}

    
  %%%% Referencias bibliográficas
  
  
    
\end{document}
