% \documentclass{article}
\usepackage{tabularray}
\usepackage{colortbl}

\begin{document}
\begin{table}
    \begin{tblr}{
        width = \linewidth,%
        colspec = {%
          X[wd=0.35\linewidth]%
          X[wd=0.15\linewidth]%
          X[wd=0.15\linewidth]%
          },
        column{2} = {r},
        column{3} = {r},
        rowhead=1,%
        row{1}={font=\bfseries},%
        row{1-5}={m},%
        cell{5}{1} = {c=3}{},
        hline{1,5} = {-}{0.08em},
        hline{2} = {-}{0.03em},
      }
      ~                                                                          & Dentro & Entre \\
      Jurisdicciones                                                             & 68,2   & 31,8  \\
      Departamentos                                                              & 46,2   & 53,8  \\
      Entornos urbanos                                                           & 23,6   & 76,4  \\
      \footnotesize{\textit{Nota:} Los valores tienen corrección de sesgo. 1.000 Bootstrap replicaciones.} &        &       
      \end{tblr}
\end{table}
\end{document}